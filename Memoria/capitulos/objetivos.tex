\chapter{Objetivos del trabajo}

Objetivos que en un inicio se pretendían abordar eran:
\begin{itemize}
	\item Introducir la rigidez y los ovaloides y demostrar la rigidez de los últimos en el resultado conocido como Teorema de Cohn-Vossen.
	\item Visualización computacional de superficies.
\end{itemize}
${ }$\\

En el primer capitulo de este trabajo se resuelve la introducción a los ovaloides y la rigidez. La demostración del teorema de Cohn-Vossen se realiza en el capítulo 3. Además en la introducción se encuentran varios resultados y conceptos necesarios para la comprensión de lo que se verán en el capítulo 3.
${ }$\\

Para la visualización de superficies primero se han creado una serie de objetos que permiten visualizar escenas en el capítulo 4. Dichos objetos son la cámara, los rayos, etc. También, se implementará un sistema de iluminación.
${ }$\\

En el capítulo 5 se aborda la visualización de superficies a partir de las ecuaciones implícitas de las mismas, las superficies que se visualizaran en este trabajo serán ovaloides.
${ }$\\