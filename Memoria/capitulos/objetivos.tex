\chapter{Objetivos del trabajo}

Objetivos que en un inicio se pretendían abordar eran:
\begin{itemize}
	\item Introducir la rigidez y los ovaloides y demostrar la rigidez de los últimos en el resultado conocido como Teorema de Cohn-Vossen.
	\item Estudio e implementación de algoritmos iterativos de cálculo de intersección entre una semirrecta (rayo) y una superficie definida por ecuaciones implícitas, con aplicación en la visualización computacional de dichas superficies en general y de ovaloides en particular.
\end{itemize}
${ }$\\

En el primer capitulo de este trabajo se resuelve la introducción a los ovaloides y la rigidez. La demostración del teorema de Cohn-Vossen se realiza en el capítulo 3. Además en la introducción se encuentran varios resultados y conceptos necesarios para la comprensión de lo que se verán en el capítulo 3.
${ }$\\

En el capítulo 4 se describe un diseño orientado a objetos que incluye las clases básicas que se suelen usar para la implementación de software de visualización por ray-tracing (cámaras, rayos, etc...). También se describe el modelo de iluminación local sencillo usual en estas aplicaciones, y su implementación. Este modelo permite cierto grado de realismo en las imágenes obtenidas, con tiempos de cálculo bajos.
${ }$\\

En el capítulo 5 se aborda la visualización de superficies a partir de las ecuaciones implícitas de las mismas, las superficies que se visualizaran en este trabajo serán ovaloides.
${ }$\\