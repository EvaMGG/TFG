\chapter{Visualización computacional. Primeros pasos.}

${ }$\\
\section{Introducción}
${ }$\\

La síntesis de imágenes por ordenador puede aplicarse a video juegos, cine, visualización científica y técnica, diseño 3d, simulación. En este proyecto nos centraremos en la visualización de ovaloides representadas por sus ecuaciones implícitas.
${ }$\\

Antes de explicar el algoritmo con el que se obtendrá la imagen de superficies mas complejas definidas con ecuaciones implícitas se comenzará implementando el sistema de rayos, cámara e intersecciones con objetos sencillos. Esto servirá de base para después generalizar a superficies mas complejas. Esta primera implementación incluirá la visualización de superficies en 3D con iluminación sencilla y sin texturas, solo con colores mate.
${ }$\\

La esfera, el cubo y el plano son ejemplos de estos objetos mas sencillos, sus soluciones pueden encontrarse mediante métodos directos, ya que puede encontrarse un fórmula analítica sencilla para hallar la intersección como se verá mas adelante en la sección \ref{cap:intRO}. Sin embargo y por lo general, encontramos superficies dadas en sus ecuaciones implícitas para las que no es posible encontrar el valor exacto de sus soluciones. En estos casos se usan métodos iterativos, que aunque son menos eficientes se pueden aplicar a casos mas generales.  En el capítulo \ref{cap:march} se verá como funcionan los métodos iterativos de Newton-Raphson y Regula-Falsi y se usará un método resultado de la combinación de ambos para la visualización de superficies.
${ }$\\




Ray-tracing en computación gráfica consiste en la generación de una imagen mediante el trazado de rayos que simulan el comportamiento de los rayos de luz al incidir en los objetos. El método de Ray-Tracing es capaz de proporcionar un grado muy alto de realismo, mas alto que otros métodos pero implica un mayor coste computacional. Por este motivo, el trazado de rayos es más adecuado para aplicaciones donde se puede tolerar un tiempo mas largo de renderización de imágenes y efectos visuales de películas y televisión y más mal adaptado para aplicaciones en tiempo real donde la velocidad es fundamental, como videojuegos o realidad virtual. El trazado de rayos es capaz de simular una gran variedad de efectos ópticos, como los fenómenos de reflexión, reflacción y dispersión (como la aberración cromática).
${ }$\\


Al menos los siguientes objetos se necesitan en la descripción de una escena (ver Figura \ref{fig:etiq_51}):
\begin{itemize}
	\item Objetos que componen la escena, con parámetros como el color del objeto.
	\item Las fuentes y posición de las fuentes de luz de la escena.
	\item Posición desde la que se visualiza la escena suele estar representado por una cámara virtual o un observador.
	\item Marco a través del cual se visualiza, es el recuadro en el que se proyectará la escena. Este marco es una rectángulo virtual que se encuentra a cierta distancia del observador. Los cuadrados de esta imagen corresponden con los píxeles de la imagen que se visualizará finalmente.
\end{itemize}
${ }$\\

\begin{figure}[h]
	\begin{center}
		\includegraphics[width=0.8\textwidth]{imagenes/tracing}
	\end{center}
	\caption{Elementos usados en Ray-tracing. Imagen procedente de \cite{ref8}}
	\label{fig:etiq_51}
\end{figure}

Para Ray-tracing usamos una estructura de datos llamada Rayo, que indica el punto de inicio, o posición del observador, y la dirección de una semirecta en el espacio. Para cada píxel, se calcula la dirección que apunta, desde el punto del ojo, al píxel correspondiente del plano de la imagen (en el rectángulo virutal). Para cada rayo, y teniendo en cuenta que el rayo puede intersecar con mas de una superficie, se calcula la distancia del observador a los los puntos intersecados por el rayo. De esta, forma nos quedamos con el objeto que se encuentra mas cerca del observador, que es el objeto que el observador puede ver en correspondencia con ese rayo. Los otros objetos estarán ocultos para el observador por este primer objeto.
${ }$\\

Ray-tracing está basado en el algoritmo denominado Ray Casting. Que es el que se encarga del trazado de rayos. Ray-tracing extiende esta idea aplicando un sombreado que tiene en cuenta efectos globales de iluminación como pueden ser reflexiones, refracciones o sombras arrojadas.
${ }$\\

Los creadores de imágenes y películas pueden tardar todo el tiempo que necesiten en renderizar un fotograma. Los videojuegos al ser interactivos necesitan obtener el siguiente fotograma en una fracción de segundo. Como resultado, la mayoría de los gráficos en tiempo real se basan en la técnica llamada rasterización. 
${ }$\\

Comparada con Ray-tracing, la rasterización es extremadamente rápida, aunque los resultados a pesar de ser muy buenos no son tan buenos como los de Ray-tracing, los cuales son imágenes que integran sombras, reflejos y refracciones de tal forma que las imágenes generadas no se pueden distinguir de fotografías o videos tomados del mundo real.
${ }$\\

Por otro lado, en el método de rasterización las imágenes se crean a partir de una malla de triángulos virtuales, que juntos forman modelos 3D de objetos. Estos modelos son proyectados en planos bidimensionales que se corresponden con la ventana de visualización, el paso final es rellenar los triángulos bidimensionales. La representación de triángulos nos lleva una mayor cantidad de información almacenada, con respecto a Ray-tracing que unicamente usa la ecuación en forma implícita para la representación de la superficie.
${ }$\\

Hemos tomado Ray-tracing por que no requiere convertir los ovaloides a mayas de triángulos, lo cual requeríría encontrar sus ecuaciones paramétricas o usar otros algoritmos muy complejos para enumerar los triángulos que forman la superficie. Por otro lado con Ray-tracing solo necesitamos su expresión en ecuaciones implícitas. Las principales fuentes han sido el libro \textit{Realistic Ray-Tracing} \cite{Shirley} y el artículo \textit{A Generalization of Algebraic Surface Drawing} \cite{Blinn}, otras fuentes han sido también el libro \textit{An Introduction to Numerical Analysis} \cite{Atkinson} y los apuntes de la asignatura \textit{Métodos Numéricos II} \cite{CarmenS}.
${ }$\\




 



${ }$\\
\section{Rayos, cámara e imagen}
${ }$\\
%%%Explicar Ray-Tracing con alguna imagen

Antes de empezar hay algunas cosas que necesitamos tener antes de empezar a programar la visualización de las superficies. Necesitaremos usar vectores para ello hay definida una clase \textbf{Tupla3f} que venía dada con el código de las practicas de \textit{Informática Gráfica} y que se encarga de almacenar vectores de 3 componentes y realizar las operaciones propias de estos objetos como son el producto escalar, producto vectorial, normalizado, etc. La forma de almacenar la información de la imagen a visualizar se llevará a cabo mediante una matriz de tantas filas y columnas como filas y columnas de pixeles tenga la ventana donde se va a visualizar la escena. Cada elemento de esta matriz almacenará el color del correspondiente pixel.
	${ }$\\	
	
El color de cada pixel se representará mediante una tupla de tres valores \textbf{double} que se corresponderán los colores rojo, azul y verde (esta representación codificación del color se llama RGB). El valor \textbf{double} que irá de 0.0 a 1.0 indicará la cantidad de ese color.
	${ }$\\	
	
Los rayos y las superficies serán representadas mediante sus expresiones matemáticas. Determinaremos el color de cada pixel mediante el cálculo de la intersección de una recta con la superficie, cada rayo se corresponderá con uno de los pixeles siendo el valor del pixel el color de la superficie del punto donde interseca el rayo (en el caso de que tenga puntos de intersección con la superficie). Para ilustrar como se corresponden los pixeles con los rayos solo basta imaginar una matriz (como un trozo de plano dividido en el espacio), frente a la escena que se quiere visualizar y hacer que de cada elemento de la matriz salga un rayo perpendicular a dicho plano. Esta es una forma de lanzar los rayos aunque la que se usará en este trabajo es diferente.
	${ }$\\	
	
En esta primera parte calcularemos la intersección del rayo con unos objetos mas sencillos de forma directa pues sabemos como calcularla. Pero hay casos en los que por la complejidad de la ecuación esto es mas complicado y necesitamos la ayuda de algoritmos de aproximación de soluciones como son Newton-Raphson y Regula-Falsi que se verán más adelante para encontrar las soluciones de las ecuaciones en forma implícita que definen la superficie.
${ }$\\
	
El color del punto se determina teniendo en cuenta el color de la superficie en dicho punto y, añadiendo la iluminación, también se considerará el ángulo que forma la fuente de iluminación con la normal del objeto en ese punto y el color de la fuente de luz como indica la Ley de Lambert.
	${ }$\\	
%%% Hablar de que están implementados los operadores con tuplas3f, producto escalar etc...


En la Figura \ref{fig:etiq_1} podemos ver como funciona una cámara estenopeica. Este mecanismo de captura de imágenes consiste en una caja cuyo interior es de color negro con un agujero en un lateral y material fotosensible que capturará la imagen en el lateral opuesto.
${ }$\\

El proceso de construcción de imagen a partir del lanzado de los rayos desde el punto del ojo es parecido a la formación de una imagen mediante una cámara estenopeica, en la que se representa un objeto en una película, como se puede ver en la Figura \ref{fig:etiq_1}. Raytracing intercambia "película" por el marco donde se proyecta la imagen y "agujero" por el punto donde se encuentra el observador. Al igual que la cámara estenopeica, la distancia entre el plano de la imagen y el punto del ojo determina la "distancia focal" y, por lo tanto, el campo de visión.
${ }$\\
	

%%% Imagen Pinhole camera

\begin{figure}[h]
	\begin{center}
		\includegraphics[width=0.6\textwidth]{imagenes/camara-estenopeica.jpg}
	\end{center}
	\caption{Cámara estenopeica. Imagen sacada de Google.}
	\label{fig:etiq_1}
\end{figure}

Primero consideraremos el sistema de referencia de visualización el cual está definido por un punto de origen \textbf{o} y una base ortonormal \{$\textbf{u}$, $\textbf{v}$, $\textbf{w}$\}. Para implementar la cámara, se tomará un rectángulo en el espacio que estará alineado con los ejes $\textbf{u}$ y $\textbf{v}$ como se muestra en la Figura\ref{fig:etiq_3}, este rectángulo se encuentra a distancia $\textbf{s}$ del origen $\textbf{o}$. Llamaremos $\textbf{a}$ y $\textbf{b}$ a las esquinas del rectángulo, que vienen dadas en coordenadas sobre la base \{$\textbf{u}$, $\textbf{v}$, $\textbf{w}$\}.
	${ }$\\

% Imagen propia 1
\begin{figure}[h]
	\begin{center}
		\includegraphics[width=1.1\textwidth]{imagenes/Imagen1.jpg}
	\end{center}
	\caption{Cálculo de los rayos de visión.}
	\label{fig:etiq_3}
\end{figure}



%%% Hacer yo una imagen con el sistema de coordenadas y el rectangulo dividido con los rayos saliendo del punto.

Sobre este rectángulo tomaremos muestras que se corresponderán con los píxeles que ocupa la ventana de visualización. Volvamos a la Figura\ref{fig:etiq_1} e imaginemos que el rectángulo está dividido en tantas filas y columnas como filas-1 y columnas-1 de pixeles necesitamos para crear la imagen. La forma de tomar el valor para estas muestras se llevará a cabo mediante rayos que pasan por $\textbf{o}$ y por las intersecciones de los segmentos que hemos usado para las divisiones que hemos trazado antes sobre el rectángulo (incluyendo los propios segmentos del rectángulo). Las coordenadas de puntos intersección de segmentos dicha se calculan mediante los siguientes cálculos, estas coordenadas están dadas en la base \{$\textbf{u}$, $\textbf{v}$\}
${ }$\\


\[
	u_i = a_1 + (b_1-a_1) \frac{i}{n_x-1}, \;\;\; 0 \leq i \leq n_x -1,
\]

\[
	v_j = a_2 + (b_2-a_2) \frac{j}{n_y-1}, \;\;\; 0 \leq j \leq n_y -1.
\]
${ }$\\

Entonces los rayos que nos interesan se expresan del siguiente modo en el sistema de referencia de visualización:
${ }$\\

\[
	p_{i,j}(t) = (0,0,0) + t(u_i , v_j, -s).
\]
${ }$\\

En las coordenadas del espacio (estas son las coordenadas en las que está expresada la ecuación de la superficie), esto se traduce en 
${ }$\\
\[
	p_{i,j}(t) = e + t( u_i \; u, v_j \; v, -s \; w).
\]
${ }$\\

En este caso, he implementado una clase $Rayo$ que se encarga de llevar las operaciones necesarias, aquí se implementa $puntoRayo$ que devuelve el punto de un rayo para un valor $t$. El código es el que se muestra a continuación:
${ }$\\

\begin{lstlisting}[style=Consola]
class Rayo{

private :

	Tupla3f origen, direccion;

public :

	Rayo(Tupla3f o, Tupla3f d);
	Tupla3f puntoRayo(double t);
	Tupla3f getOrigen();
	Tupla3f getDireccion();
};
\end{lstlisting}
${ }$\\

\begin{lstlisting}[style=Consola]
Rayo::Rayo(Tupla3f o, Tupla3f d){

	origen = o;
	direccion = d;
}

Tupla3f Rayo::puntoRayo(double t){

	return origen + t*direccion;
}

Tupla3f Rayo::getOrigen(){

	return origen;
}

Tupla3f Rayo::getDireccion(){

	return direccion;
}
\end{lstlisting}
${ }$\\

Para poder establecer el punto \textbf{o} de visión y la orientación se calcula el sistema de referencia de visión del siguiente modo:
\begin{itemize}
	\item \textbf{o} : Será el que hemos visto hasta ahora, el punto de visión (donde se encuentra el observador) dado en coordenadas del espacio $xyz$.
	\item \textbf{g} : Se define como un vector en las coordenadas $xyz$ que indica la dirección en la que el observador mira.
	\item \textbf{$v_{up}$} : es un vector que apunta hacia arriba (hacia donde nosotros queramos que sea arriba en la imagen).
	\item \textbf{s} : la distancia de \textbf{o} al rectángulo de visión.
	\item \textbf{a}, \textbf{b} : los que hemos visto con anterioridad, son las esquinas del rectángulo de visión en la base \{$\textbf{u}$, $\textbf{v}$, $\textbf{w}$\}
\end{itemize}
${ }$\\

Con estos datos los vectores de la base ortonormal de visión son
${ }$\\
	\[
		w = - \frac{g}{\parallel g \parallel},
	\]
	\[
		u = \frac{v_{up} \times w}{\parallel v_{up} \times w \parallel},
	\]
	
	\[
		v = w \times u.
	\]
	${ }$\\
	
Si queremos que el rectángulo de visión este centrado con respecto a g hay que asegurarse de que $a_1 = -b_1$ y $a_2 = -b_2$. Además, si no queremos que la imagen resultante se vea estirada en alguno de los ejes \textbf{u} o \textbf{v} debemos asegurarnos de las que divisiones realizadas sobre el rectángulo de visión sean del mismo tamaño tanto para las filas como para las columnas quedando dividido el rectángulo en cuadrados, esto se consigue asegurándose de que
${ }$\\
	\[
		\frac{b_1-a_1}{b_2 - a_2} = \frac{n_x}{n_y}.
	\]
	${ }$\\

En el siguiente código la función $Image$ se encarga de devolver la imagen que se ve desde el punto del observador en una matriz de dimensiones iguales a las del tamaño de la ventana donde se muestra la imagen, la función $interposicionObjeto$ y la Ley de Lambert se verán mas adelante en la parte de iluminación. La función $Inicializar$ se encarga de inicializar los objetos de la escena y otras variables necesarias para la visualización.
${ }$\\

\begin{lstlisting}[style=Consola]
vector<Superficie *> superficies;

double a1, a2, b1, b2;
double s = 6.0;
Tupla3f e = Tupla3f(0,0,8);
Tupla3f g = Tupla3f(0,0,-2);
Tupla3f vup = Tupla3f(0,1,0);
Tupla3f u, v, w;
Tupla3f *image;

LuzDireccional luz = LuzDireccional(Tupla3f(1.5,1.5,1.65), Tupla3f(1, 1, 1));


void Inicializar(int x, int y) {

	superficies.push_back(new Esfera(Tupla3f(0,0,0), 2.0, Tupla3f(0.1, 0.5, 0.1), Tupla3f(0.25,0.4,0.2), 1.5));
	
	superficies.push_back(new Cubo(Tupla3f(1.5,1.5,1.5), Tupla3f(2,2,2), Tupla3f(0.1, 0.1, 0.5), Tupla3f(0.2,0.1,0.3), 0.8));
	
	superficies.push_back(new Esfera(Tupla3f(-1.5,-1.5,1.5), 0.5, Tupla3f(0.5, 0.1, 0.1), Tupla3f(0.3,0.1,0.1), 2.0));
	
	image = new Tupla3f [x*y];

	a1=-x/200.0;
	a2=y/200.0;
	b1=-a1;
	b2=-a2;

	w = (-1)*normalized(g);
	u = normalized(vup*w);
	v = w*u;
}

\end{lstlisting}
${ }$\\


\begin{lstlisting}[style=Consola]
Tupla3f* Image(int x, int y){

	Tupla3f direccion;
	double int_ant;

	for(int i = 0; i < x; i++) {
	
		for (int j = 0; j < y; j++){
		
			direccion = (a1 + (b1 - a1)*(i/(x-1.0)))*u + (a2 + (b2 - a2)*(j/(y-1.0)))*v -s*w;
			
			Tupla3f e = direccion / len(direccion);
			Rayo rayo = Rayo(o, direccion);

			if (!superficies.empty()){

				int indice_min;

				primeraInterseccion(rayo, int_ant, indice_min);
				Superficie *priSup = (superficies[indice_min]);

				if (int_ant >= 0) {

					image[i*y +j] = fuentesLuz[0]->colorPixel( priSup->normal(rayo.puntoRayo(int_ant)), e, priSup->getEspecular(), priSup->getBrillo(), priSup->getColor(), rayo.puntoRayo(int_ant));

					if (interposicionSuperficie(rayo, int_ant, indice_min) == true) image[i*y +j] = Tupla3f(0, 0, 0);
				}
				
				else image[i*y +j] = Tupla3f(1, 1, 1);
			}
		}
	}
	
	return image;
}


\end{lstlisting}
${ }$\\

Cuando hay varios objetos en la escena hay que tener en cuenta cual es el primero que se encuentra el rayo en su recorrido, pues los que encuentre con posterioridad no serán vistos por el observador. Esto es de lo que se encarga la siguiente función, esto es, para cada objeto calcula donde interseca con un rayo dado y devuelve el objeto con el que se encuentra primero y el lugar del rayo donde interseca.
${ }$\\

\begin{lstlisting}[style=Consola]

void primeraInterseccion(Rayo rayo, double &min_inter, int &indice_min){

	indice_min = 0;
	min_inter = (superficies[0])->interseccion(rayo);

	for (int k = 1; k < (int)superficies.size(); k++) {
	
		double intersec = (superficies[k])->interseccion(rayo);

		if ( (intersec < min_inter && intersec >= 0) || (min_inter == -1 && intersec >= 0) ) {
		
			indice_min = k;
			min_inter = intersec;
		}
	}
}
\end{lstlisting}
${ }$\\

${ }$\\
\section{Intersección de rayos con objetos}\label{cap:intRO}
${ }$\\

Vamos a describir algunos algoritmos sencillos directos de cálculo de intersecciones rayo-objeto, para objetos que tienen asociada una fórmula analítica sencilla que nos permite el calculo directo de esas intersecciones. De entre la superficies que podemos encontrar con esta característica se veran solo el caso de la esfera y el del cubo.
${ }$\\

Las clases $Esfera$ y $Cubo$ que se van definir son subclase de la clase $Superficie$ definida como sigue:
${ }$\\

\begin{lstlisting}[style=Consola]
class Superficie {

private :

	Tupla3f color;
	Tupla3f compEspecular;
	double expBrillo;

public :

	Superficie(Tupla3f col, Tupla3f comEs, double expB);
	virtual double interseccion(Rayo rayo)=0;
	virtual Tupla3f normal(Tupla3f xyz)=0;
	virtual Tupla3f getColor();
	virtual Tupla3f getEspecular();
	virtual double getBrillo();
};
\end{lstlisting}
${ }$\\

\begin{lstlisting}[style=Consola]
Superficie::Superficie(Tupla3f col, Tupla3f comEs, double expB){
color = col;
compEspecular = comEs;
expBrillo = expB;
}

Tupla3f Superficie::getColor(){
return color;
}

Tupla3f Superficie::getEspecular(){
return compEspecular;
}

double Superficie::getBrillo(){
return expBrillo;
}

\end{lstlisting}
${ }$\\

${ }$\\
\subsection{Intersección rayo-esfera}
%$\textbf{4.3.1. Intersección rayo-esfera}$
${ }$\\

Una esfera con centro en $c = (c_x, c_y, c_z)$ y radio $R>0$ se puede representar mediante la siguiente ecuación:
${ }$\\
\[
	(x-c_x)^2 + (y-c_y)^2 + (z-c_z)^2 = 0,
\]
${ }$\\
también se puede expresar como sigue
${ }$\\
\[
	(p-c)\cdot(p-c) - R^2 = 0,
\]
${ }$\\
donde cualquier punto que cumpla la ecuación está en la esfera. En nuestro caso queremos estudiar la intersección de la superficie y un rayo dado por $p(t) = o + td$. Para encontrar los puntos de intersección del rayo con la esfera evaluamos esta expresión en $p(t)$,
${ }$\\
\[
	(o+td-c)\cdot(o+td-c) - R^2 = 0,
\]
${ }$\\
si desarrollamos un poco, obtenemos
${ }$\\
\[
	(d\cdot d)t^2 + 2d\cdot (o-c)t + (o-c)\cdot(o-c) - R^2 = 0,
\]
${ }$\\
podemos observar que esta es una ecuación de la forma $At^2+Bt+C=0$ de la cual podemos conocer sus raíces, que como ya sabemos son
${ }$\\
\[
	t = \frac{-B\pm \sqrt{B^2-4CA}}{2A},
\]
${ }$\\
aplicando esto a nuestro caso, tenemos
${ }$\\
\[
	t = \frac{-d\cdot (o-c) \pm \sqrt{(d\cdot (o-c))^2 - (d\cdot d)((o-c)\cdot(o-c)-R^2)}}{d\cdot d},
\]
${ }$\\
donde como ya hemos visto $t$ indica el lugar del rayo donde alcanza la superficie.
${ }$\\

Para saber si el rayo interseca con la superficie se tendrá en cuenta el signo del discriminante $B^2-4CA$. En el caso de que el discriminante sea igual a cero solo habrá una solución. Si fuese positivo el rayo interseca con la superficie en dos puntos (uno de ellos es el punto a través del cual entra en la superficie y el otra por donde sale de la esfera). Por último, si dicho valor es negativo el valor de la raíz será imaginario y no habrá puntos de intersección de la esfera con el rayo.
${ }$\\
	
A continuación se muestra el código de la clase Esfera la cual permite la creación de una esfera con centro y radio elegidos. Además contiene un método que nos devuelve la distancia del punto origen $\textbf{o}$, de un rayo pasado como parámetro, al punto de intersección mas cercano de la esfera.
${ }$\\

\begin{lstlisting}[style=Consola]
class Esfera : public Superficie {

private :

	Tupla3f centro;
	double radio;

public :

	Esfera(Tupla3f c, double r, Tupla3f col, Tupla3f comEs, double expB);
	double interseccion(Rayo rayo);
	Tupla3f normal(Tupla3f xyz);
};
\end{lstlisting}
${ }$\\

\begin{lstlisting}[style=Consola]
Esfera::Esfera(Tupla3f c, double r, Tupla3f col, Tupla3f comEs, double expB):Superficie(col, comEs, expB) {

	centro = c;
	radio = r;
}

double Esfera::interseccion(Rayo rayo){

	double interseccion = -1;

	double discriminante = (rayo.getDireccion()|(rayo.getOrigen()-centro))*(rayo.getDireccion()|(rayo.getOrigen()-centro)) - (rayo.getDireccion()|rayo.getDireccion())*(((rayo.getOrigen()-centro)|(rayo.getOrigen()-centro)) - radio*radio);

	if (discriminante > 0) {
	
		double t0 = ( ( -(rayo.getDireccion()|(rayo.getOrigen()-centro)) + sqrt( discriminante ) ) / ( rayo.getDireccion()|rayo.getDireccion() ) );
		
		double t1 = ( ( -(rayo.getDireccion()|(rayo.getOrigen()-centro)) - sqrt( discriminante ) ) / ( rayo.getDireccion()|rayo.getDireccion() ) );

		if (t0 > 0) interseccion = t0;
		if (t0 > t1 && t1>0) interseccion = t1;
	}
	
	return interseccion;
}


Tupla3f Esfera::normal(Tupla3f xyz) {

	return Tupla3f( (xyz-centro)/radio );
}
\end{lstlisting}
${ }$\\

${ }$\\
\subsection{Intersección rayo-caja}
%$\textbf{4.3.2. Inerseccion rayo-caja}$
${ }$\\

Las cajas de esta sección serán representadas mediante las coordenadas de dos esquinas opuestas $p_0 = (x_0, y_0, z_0)$ y $p_1 = (x_1, y_1, z_1)$ y los lados de dichas cajas estarán alineados con los ejes. Para ver la intersección de los rayos usaremos un método que resulta ser más rápido que comprobar la intersección con cada una de las seis caras y que es la que aparece en el libro \cite{Shirley}.
	${ }$\\	
	
El método al que me refiero considera tres intervalos que se corresponden con cada una de las tres coordenadas, así un punto $(x, y, z)$ que esté dentro del cubo debe cumplir $x \in [x_0, x_1]$, $y \in [y_0, y_1]$ y $z \in [z_0, z_1]$. De este modo, para un rayo dado por $p(t) = o + td$ calcularemos en que intervalo ha de encontrarse $t$ para que el rayo corte la superficie. Para $x \in [x_0, x_1]$,
${ }$\\
\[
	x_0 = o_x + t_{x0}d_x
\]
\[
	x_1 = o_x + t_{x1}d_x
\]
${ }$\\
y despejando $t_{xi}$ en cada ecuación obtenemos que el segmento de rayo que está dentro del cubo se corresponde con
${ }$\\
\[
	t \in [(x_0-o_x)/d_x, (x_1-o_x)/d_x]
\]
${ }$\\

Para definir el intervalo de $t's$ para los que el rayo está en la franja de espacio $[x_0, x_1]$ cuenta si $d_x$ es positivo o negativo, en caso de que sea positivo el intervalo sería el esperado $[t_{x0}, t_{x1}]$, pero en caso de ser negativo sería de la forma $[t_{x1}, t_{x0}]$, por comodidad este intervalo se notará como $[t_{x min}, t_{x max}]$.
	${ }$\\	
% Imagen propia 2
%${ }$\\
\begin{figure}[h]
	\begin{center}
		\includegraphics[width=0.8\textwidth]{imagenes/Imagen2.jpg}
	\end{center}
	\caption{Cálculo del intervalo de t's para los cuales el rayo está en la superficie.}
	\label{fig:etiq_4}
\end{figure}




El proceso seguido anteriormente para el intervalo en $x$, será el seguido para el intervalo en $y$ y $z$ obteniendo $[t_{y min}, t_{y max}]$ y $[t_{z min}, t_{z max}]$. Una vez calculados los tres intervalos se comprobará si el rayo corta la superficie y cual es el primer punto donde lo hace, esto se hará calculando la intersección de los tres intervalos y comprobando si es vacío o no y en caso de que no sea vacío se devolverá el mínimo del intervalo. Para esto último el algoritmo implementado tomará $t_{min} = max\{t_{x min}, t_{y min}, t_{z min}\}$ y $t_{max} = min\{t_{x max}, t_{y max}, t_{z max}\}$, siendo el intervalo $[t_{min}, t_{max}]$ el intervalo de los $t$ para los que el rayo se encuentra dentro del cubo. Si $t_{min} > t_{max}$ entonces el intervalo es vacío y el rayo no toca al cubo, en caso contrario el algoritmo calcula en qué lugar del rayo se corta por primera vez la superficie, lo que quiere decir que, nos devolverá $t_{min}$. Para ilustrar este método se muestra una simplificación del problema en el espacio 2-dimensional en la Figura\ref{fig:etiq_4}.
${ }$\\	



El código que corresponde a la representación computacional del cubo es el que se muestra a continuación:
${ }$\\

\begin{lstlisting}[style=Consola]
class Cubo : public Superficie {

private :

	Tupla3f esquina1, esquina2;

public :

	Cubo(Tupla3f e1, Tupla3f e2, Tupla3f col, Tupla3f comEs, double expB);
	double interseccion(Rayo rayo);
	Tupla3f normal(Tupla3f xyz);
};
\end{lstlisting}
${ }$\\

\begin{lstlisting}[style=Consola]
Cubo::Cubo(Tupla3f e1, Tupla3f e2, Tupla3f col, Tupla3f comEs, double expB):Superficie(col, comEs, expB) {

	esquina1 = e1;
	esquina2 = e2;
}

double Cubo::interseccion(Rayo rayo){

	double tx_min, tx_max, ty_min, ty_max, tz_min, tz_max, t0, t1;

	if (rayo.getDireccion().coo[0] > 0) {
	
		tx_min = (esquina1.coo[0] - rayo.getOrigen().coo[0])/rayo.getDireccion().coo[0];
		
		tx_max = (esquina2.coo[0] - rayo.getOrigen().coo[0])/rayo.getDireccion().coo[0];
	}
	
	else {
	
		tx_min = (esquina2.coo[0] - rayo.getOrigen().coo[0])/rayo.getDireccion().coo[0];
		
		tx_max = (esquina1.coo[0] - rayo.getOrigen().coo[0])/rayo.getDireccion().coo[0];
	}

	if (rayo.getDireccion().coo[1] > 0) {
	
		ty_min = (esquina1.coo[1] - rayo.getOrigen().coo[1])/rayo.getDireccion().coo[1];
		
		ty_max = (esquina2.coo[1] - rayo.getOrigen().coo[1])/rayo.getDireccion().coo[1];
	}
	
	else {
	
		ty_min = (esquina2.coo[1] - rayo.getOrigen().coo[1])/rayo.getDireccion().coo[1];
		
		ty_max = (esquina1.coo[1] - rayo.getOrigen().coo[1])/rayo.getDireccion().coo[1];
	}

	if (rayo.getDireccion().coo[2] > 0) {
	
		tz_min = (esquina1.coo[2] - rayo.getOrigen().coo[2])/rayo.getDireccion().coo[2];
		
		tz_max = (esquina2.coo[2] - rayo.getOrigen().coo[2])/rayo.getDireccion().coo[2];
	}
	
	else {
	
		tz_min = (esquina2.coo[2] - rayo.getOrigen().coo[2])/rayo.getDireccion().coo[2];
		
		tz_max = (esquina1.coo[2] - rayo.getOrigen().coo[2])/rayo.getDireccion().coo[2];
	}

	if (tx_min > ty_min) t0 = tx_min;
	else t0 = ty_min;
	if (tz_min > t0) t0 = tz_min;

	if (tx_max < ty_max) t1 = tx_max;
	else t1 = ty_max;
	if (tz_max < t1) t1 = tz_max;


	if (t0 < t1) return t0;
	else return -1;
}


Tupla3f Cubo::normal(Tupla3f xyz) {

	if ( abs(xyz.coo[0] - esquina2.coo[0]) <= 0.01 ) return Tupla3f(1,0,0);
	
	else if ( abs(xyz.coo[1] - esquina2.coo[1]) <= 0.01 ) return Tupla3f(0,1,0);
	
	else if ( abs(xyz.coo[2] - esquina2.coo[2]) <= 0.01 ) return Tupla3f(0,0,1);
	
	else if ( abs(xyz.coo[0] - esquina1.coo[0]) <= 0.01 ) return Tupla3f(-1,0,0);
	
	else if ( abs(xyz.coo[1] - esquina1.coo[1]) <= 0.01 ) return Tupla3f(0,-1,0);
	
	else if ( abs(xyz.coo[2] - esquina1.coo[2]) <= 0.01 ) return Tupla3f(0,0,-1);
}
\end{lstlisting}
${ }$\\

${ }$\\
\section{Iluminación}
%$\textbf{4.4. ILUMINACIÓN}$
${ }$\\

La iluminación hace las imágenes más realistas y nos permite apreciar la profundidad de los objetos. En esta sección veremos una implementación de la iluminación sencilla que solo dependerá de la normal de la superficie y la dirección de incidencia de la fuente de iluminación. 
${ }$\\

Como se muestra en la Figura\ref{fig:etiq_5}, el rayo de visión interseca la superficie en un punto iluminado, este rayo esta definido por el vector \textbf{d} y a partir de él definimos el vector unidad \textbf{e} el cual tiene la misma dirección de \textbf{d} pero sentido opuesto:
${ }$\\

%%% Hacer una imagen como la del libro pero que sea propia

\[
	\textbf{e} = - \frac{\textbf{d}}{\|\textbf{d}\|},
\]
${ }$\\
este vector será prescindible a la hora de añadir la iluminación, se utiliza para materiales que cambian su brillo de posición cuando el observador cambia también la suya.
${ }$\\


El vector \textbf{l} indica la dirección desde la que llega la luz y \textbf{n} el vector normal a la superficie. Con todo esto la Ley de Lambert sería
${ }$\\
\[
	L = ER(\textbf{n}\cdot \textbf{l}),
\]
${ }$\\
esta ley nos da el valor del color del pixel habiendo añadido la iluminación. En esta fórmula $E$ es el valor del color de la fuente de luz y $R$ el color del punto de incidencia, el producto de ellos es componente a componente. Debemos observar que $L$ puede ser negativo, cuando esto ocurre podemos tomar dos caminos, uno de los cuales es hacer cero los valores negativos, este es el que se va a usar aquí es el que vamos a usar, y otro consiste en tomar el valor absoluto.
	${ }$\\	


%Imagen propia 3
\begin{figure}[h]
	\begin{center}
		\includegraphics[width=0.7\textwidth]{imagenes/Imagen3.jpg}
	\end{center}
	\caption{Elementos geométricos para la implementación de la luz.}
	\label{fig:etiq_5}
\end{figure}

La componente especular se encarga de dotar a las superficies de brillo. Para programarla hemos tomado la componente especular del modelo de Phong que viene en los apuntes \cite{carlosU}. Dicha componente se calcula como:
${ }$\\

$$ sumandoEspecular(p, e, l) \; = \; compEspecular \; * \; d \; * \; [max(0, r \cdot e)]^{expBrillo}.$$
${ }$\\

Donde, $d$ es $1$ si $l	\cdot n > 0$ y en caso contrario $0$, y $r$ es el vector reflejado de $l$ a con respecto a $n$ y se calcula del siguiente modo:
${ }$\\

$$r = 2 \; * \; (l \cdot n) \; * \; n - l. $$
${ }$\\


Hasta ahora solo se ha visto como implementar sombras con un solo objeto pero cuando hay más objetos en la escena hay sombras que se producen cuando un objeto se interpone entre la fuente de luz y otro objeto. Las sombras creadas de este modo son llamadas sobras arrojadas.
	${ }$\\	
	
Para simular las sombras arrojadas se lanzan rayos sombra, estos rayos van desde el punto de intersección hasta las fuentes de luz. En este caso tomamos un rayo que sale de un punto \textbf{q} en dirección \textbf{l} pero sentido opuesto. Si el rayo interseca con otra superficie el valor de $L$ se pone a cero en caso contrario se calcula el valor del color mediante la Ley de Lambert. Esto se ilustra en la Figura\ref{fig:etiq_6}. No se tendrán en cuenta objetos que intersequen en un valor de t negativo o cero.
	${ }$\\	
	
Puede ocurrir que por razones de precisión finita haya puntos que se oscurezcan por que se haya calculado que el rayo de sombra interseca a la propia superficie (la que contiene a q), para evitar este problema lo que hacemos es considerar solo las intersecciones que tienen un $t > \epsilon$ para $\epsilon > 0 $.
	${ }$\\	

\begin{figure}[h]
	\begin{center}
		\includegraphics[width=1.0\textwidth]{imagenes/Imagen4.jpg}
	\end{center}
	\caption{Implementación de sombras para mas de un objeto. El punto en rojo se coresponde con un punto al que no le da la luz por que la intercepta otro objeto. En cambio el punto en azul es alcanzado por la luz ya que no hay objetos que se la tapen.}
	\label{fig:etiq_6}
\end{figure}

El siguiente código implementa la fuente de luz que solo tiene como atributos la dirección con la que la luz incide en los objetos y el color de la misma. Tambien se ha creado una clase $Iluminacion$ de la $LuzDireccional$ hereda. El código que implementa la iluminación se ha mostrado con anterioridad en la función $Image$. A continuación se muestra el código de la función $interposicionObjeto$ el cual es llamado desde la función $Image$ que ya se ha visto anteriormente. Está función se encarga, para cada rayo, de comprobar si (para cada punto visible para el observador) hay una superficie que se interpone entre la luz y dicho punto del objeto para hacer que ese punto se vea sombreado. Devuelte true si hay una superficie que se interpone o falso en caso contrario.
	${ }$\\



\begin{lstlisting}[style=Consola]
class Iluminacion{

private:

	Tupla3f color;

public:

	Iluminacion(Tupla3f col);
	virtual Tupla3f LeyLambert(Tupla3f objColor, Tupla3f normal, Tupla3f punto);
	virtual Tupla3f getDireccion(Tupla3f punto)=0;
	virtual Tupla3f colorPixel(Tupla3f normal, Tupla3f e, Tupla3f compE, double brillo, Tupla3f compA, Tupla3f compD)=0;
};
\end{lstlisting}
${ }$\\

\begin{lstlisting}[style=Consola]
Iluminacion::Iluminacion(Tupla3f col){

	color = col;
}

Tupla3f Iluminacion::LeyLambert(Tupla3f objColor, Tupla3f normal, Tupla3f punto) {

	Tupla3f L = Tupla3f(color.coo[0]*objColor.coo[0], color.coo[1]*objColor.coo[1], color.coo[2]*objColor.coo[2]) * (normal | getDireccion(punto));
	
	return L;
}
\end{lstlisting}
${ }$\\


\begin{lstlisting}[style=Consola]

class LuzDireccional : public Iluminacion {

private:

	Tupla3f direccion;

public:

	LuzDireccional(Tupla3f d, Tupla3f col);
	Tupla3f getDireccion(Tupla3f punto);
	Tupla3f colorPixel(Tupla3f normal, Tupla3f e, Tupla3f compE, double brillo, Tupla3f color, Tupla3f punto);
};
\end{lstlisting}
${ }$\\

\begin{lstlisting}[style=Consola]
LuzDireccional::LuzDireccional(Tupla3f d, Tupla3f col):Iluminacion(col) {

	direccion = d;
}

Tupla3f LuzDireccional::getDireccion(Tupla3f punto) {

	return direccion;
}

Tupla3f LuzDireccional::colorPixel(Tupla3f normal, Tupla3f e, Tupla3f compE, double brillo, Tupla3f col, Tupla3f punto){

	bool d =((normal|direccion)>0);
	Tupla3f reflejado = 2*(direccion|normal)*normal-direccion;

	Tupla3f sumandoEspecular = compE*d*pow(max((float) 0, (e|(-1)*reflejado)),brillo);
	
	Tupla3f color = LeyLambert(col, normal, punto) + sumandoEspecular;

	return color;
}


\end{lstlisting}
${ }$\\

\begin{lstlisting}[style=Consola]

bool interposicionSuperficie(Rayo rayo, double min_inter, int indice_min){

	bool inter = false;
	int k=0;
	
	while ( (k < (int)superficies.size()) && !inter) {

		if ( ((superficies[k])->interseccion(Rayo(rayo.puntoRayo(min_inter), fuentesLuz[0]->getDireccion(rayo.puntoRayo(min_inter))) ) >= 0.0) && (k!= indice_min) )  inter = true;
	
		k++;
	}
	
	return inter;
}
\end{lstlisting}
	${ }$\\
	
Finalmente para comprobar que lo que hemos programado hasta ahora funciona correctamente he realizado una prueba usando los objetos que creé y almacené en el vector $superficies$ que se ve en el código presentado al principio en la función $Inicializar$. Después de ejecutarlo el resultado fue el que se muestra en la Figura \ref{fig:etiq_9}.
${ }$\\

\begin{figure}[h]
	\begin{center}
		\includegraphics[width=0.8\textwidth]{imagenes/prueba.png}
	\end{center}
	\caption{Ejemplo de comprobación con luz direccional.}
	\label{fig:etiq_9}
\end{figure}
	
Adicionalmente, la clase luz puntual, la cual almacena el punto donde está la fuente de luz y desde la que salen los rayos de luz en todas direcciones.
${ }$\\

\begin{lstlisting}[style=Consola]
class LuzPuntual : public Iluminacion {

private:

	Tupla3f posicion;

public:

	LuzPuntual(Tupla3f p, Tupla3f col);
	Tupla3f getDireccion(Tupla3f punto);
	Tupla3f colorPixel(Tupla3f normal, Tupla3f e, Tupla3f compE, double brillo, Tupla3f col, Tupla3f compD);
};
\end{lstlisting}
${ }$\\

\begin{lstlisting}[style=Consola]
LuzPuntual::LuzPuntual(Tupla3f p, Tupla3f col):Iluminacion(col) {

	posicion = p;
}

Tupla3f LuzPuntual::getDireccion(Tupla3f punto) {

	return normalized(punto - posicion);
}

Tupla3f LuzPuntual::colorPixel(Tupla3f normal, Tupla3f e, Tupla3f compE, double brillo, Tupla3f col, Tupla3f punto){

	bool d =((normal|(-1)*getDireccion(punto))>0);
	
	Tupla3f reflejado = 2*((-1)*getDireccion(punto)|normal)*normal-(-1)*getDireccion(punto);

	Tupla3f sumandoEspecular = compE*d*pow(max((float) 0, (e|reflejado)),brillo);

	Tupla3f color = LeyLambert(col, normal, punto) + sumandoEspecular;

	return color;
}
\end{lstlisting}
${ }$\\

El resultado de usar la luz puntual se muestra en la Figura \ref{fig:etiq_11}.
${ }$\\

\begin{figure}[h]
	\begin{center}
		\includegraphics[width=0.8\textwidth]{imagenes/puntual.png}
	\end{center}
	\caption{Ejemplo de luz puntual.}
	\label{fig:etiq_11}
\end{figure}

