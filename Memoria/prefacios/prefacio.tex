\chapter*{}
%\thispagestyle{empty}
%\cleardoublepage

%\thispagestyle{empty}

\begin{titlepage}
 
 
\setlength{\centeroffset}{-0.5\oddsidemargin}
\addtolength{\centeroffset}{0.5\evensidemargin}
\thispagestyle{empty}

\noindent\hspace*{\centeroffset}\begin{minipage}{\textwidth}

\centering
%\includegraphics[width=0.9\textwidth]{imagenes/logo_ugr.jpg}\\[1.4cm]

%\textsc{ \Large PROYECTO FIN DE CARRERA\\[0.2cm]}
%\textsc{ INGENIERÍA EN INFORMÁTICA}\\[1cm]
% Upper part of the page
% 

 \vspace{3.3cm}

%si el proyecto tiene logo poner aquí
\includegraphics[width=0.3\textwidth]{imagenes/logougr_new.png}\\[1.4cm]

 \vspace{0.5cm}

% Title

{\Huge\bfseries Rigidez de ovaloides : Teorema de Cohn-Vossen. Visualización computacional\\
}
%%%%\noindent\rule[-1ex]{\textwidth}{3pt}\\[3.5ex]
%%%%{\large\bfseries Subtítulo del proyecto.\\[4cm]}
\end{minipage}
${ }$\\
${ }$\\
${ }$\\

\vspace{2.5cm}
\noindent\hspace*{\centeroffset}\begin{minipage}{\textwidth}
\centering

\textbf{Autor}\\ {Eva María González García}\\[2.5ex]

\textbf{Tutores}\\
\begin{multicols}{2}
	Francisco Urbano Pérez-Aranda\\
	\columnbreak
	Carlos Ureña Almagro
\end{multicols}%%}\\[2cm]
%\includegraphics[width=0.15\textwidth]{imagenes/tstc.png}\\[0.1cm]
%\textsc{Departamento de Teoría de la Señal, Telemática y Comunicaciones}\\
%\textsc{---}\\
%Granada, mes de 201
\end{minipage}
%\addtolength{\textwidth}{\centeroffset}
\vspace{\stretch{2}}

 
\end{titlepage}






\cleardoublepage
\thispagestyle{empty}

\begin{center}
{\large\bfseries Rigidez de ovaloides y visualización computacional}\\
\end{center}
\begin{center}
Eva María González García\\
\end{center}


\vspace{0.7cm}
\noindent{\textbf{Resumen}}\\

En este trabajo recordaremos que eran los ovaloides y otros resultados y definiciones como el de movimiento rígido e isometría. Veremos que son las superficies rígidas y llegaremos a la conclusión de que los ovaloides son superficies rígidas. También se verá un algoritmo de visualización como es el Ray-marching el cual mediante rayos lazados desde algún punto y algoritmos iterativos de aproximación de soluciones como son Newton-Raphson y Regula-Falsi nos permitirá visualizar superficies representadas de forma computacional mediante su ecuación implícita y visualizaremos algunos ejemplos de ovaloides mediante este procedimiento.


\vspace{0.7cm}
\noindent{\textbf{Palabras clave}: superficie rígida, ovaloides, movimiento rígido, isometría, ray-marching, newton-raphson, regula-falsi.}\\
\cleardoublepage


\thispagestyle{empty}


\begin{center}
{\large\bfseries Project Title: Project Subtitle}\\
\end{center}
\begin{center}
Eva María González García\\
\end{center}

\vspace{0.7cm}
\noindent{\textbf{Abstract}}\\

Write here the abstract in English.

Definición intuitiva de superficie de montiel y ros pag 34.

Cuvatura de Gauss y difenrecial del normal.

Movimientos rígidos e isometrías.

Superficies rígidas.

Ovaloides y rigidez de ovaloides.

\vspace{0.7cm}
\noindent{\textbf{Keywords}: Keyword1, Keyword2, Keyword3, ....}\\

\chapter*{}
\thispagestyle{empty}

\noindent\rule[-1ex]{\textwidth}{2pt}\\[4.5ex]

Yo, \textbf{Eva Mª González García}, alumno de la titulación DOBLE GRADO EN INGENIERÍA INFORMÁTICA Y MATEMÁTICAS de la \textbf{Escuela Técnica Superior
de Ingenierías Informática y de Telecomunicación de la Universidad de Granada}y la \textbf{Facultad de ciencias de la Universidad de Granada}, con DNI XXXXXXXXX, autorizo la
ubicación de la siguiente copia de mi Trabajo Fin de Grado en la biblioteca del centro para que pueda ser
consultada por las personas que lo deseen.

\vspace{6cm}

\noindent Fdo: Eva Mª González García

\vspace{2cm}

\begin{flushright}
Granada a X de mes de 2018.
\end{flushright}


\chapter*{}
\thispagestyle{empty}

\noindent\rule[-1ex]{\textwidth}{2pt}\\[4.5ex]

D. \textbf{Francisco Urbano Pérez-Aranda}, Profesor del Área de XXXX del Departamento YYYY de la Universidad de Granada.

\vspace{0.5cm}

D. \textbf{Carlos Ureña Almagro}, Profesor del Área de XXXX del Departamento YYYY de la Universidad de Granada.


\vspace{0.5cm}

\textbf{Informan:}

\vspace{0.5cm}

Que el presente trabajo, titulado \textit{\textbf{Rigidez de ovaloides y visualización computacional}},
ha sido realizado bajo su supervisión por \textbf{Eva Mª González García}, y autorizamos la defensa de dicho trabajo ante el tribunal que corresponda.

\vspace{0.5cm}

Y para que conste, expiden y firman el presente informe en Granada a X de mes de 2018 .

\vspace{1cm}

\textbf{Los tutores:}

\vspace{5cm}

\noindent \textbf{Francisco Urbano Pérez-Aranda \ \ \ \ \ Carlos Ureña Almagro}

%\chapter*{Agradecimientos}
%\thispagestyle{empty}

%       \vspace{1cm}


%Poner aquí agradecimientos...

