\chapter*{}
%\thispagestyle{empty}
%\cleardoublepage

%\thispagestyle{empty}

\begin{titlepage}
 
 
\setlength{\centeroffset}{-0.5\oddsidemargin}
\addtolength{\centeroffset}{0.5\evensidemargin}
\thispagestyle{empty}

\noindent\hspace*{\centeroffset}\begin{minipage}{\textwidth}

\centering
%\includegraphics[width=0.9\textwidth]{imagenes/logo_ugr.jpg}\\[1.4cm]

%\textsc{ \Large PROYECTO FIN DE CARRERA\\[0.2cm]}
%\textsc{ INGENIERÍA EN INFORMÁTICA}\\[1cm]
% Upper part of the page
% 

 \vspace{3.3cm}

%si el proyecto tiene logo poner aquí
\includegraphics[width=0.3\textwidth]{imagenes/logougr_new.png}\\[1.4cm]

 \vspace{0.5cm}

% Title

{\Huge\bfseries Rigidez de ovaloides : Teorema de Cohn-Vossen. Visualización computacional\\
}
%%%%\noindent\rule[-1ex]{\textwidth}{3pt}\\[3.5ex]
%%%%{\large\bfseries Subtítulo del proyecto.\\[4cm]}
\end{minipage}
${ }$\\
${ }$\\
${ }$\\

\vspace{2.5cm}
\noindent\hspace*{\centeroffset}\begin{minipage}{\textwidth}
\centering

\textbf{Autor}\\ {Eva María González García}\\[2.5ex]

\textbf{Tutores}\\
\begin{multicols}{2}
	Francisco Urbano Pérez-Aranda\\
	\columnbreak
	Carlos Ureña Almagro
\end{multicols}%%}\\[2cm]
%\includegraphics[width=0.15\textwidth]{imagenes/tstc.png}\\[0.1cm]
%\textsc{Departamento de Teoría de la Señal, Telemática y Comunicaciones}\\
%\textsc{---}\\
%Granada, mes de 201
\end{minipage}
%\addtolength{\textwidth}{\centeroffset}
\vspace{\stretch{2}}

 
\end{titlepage}






\cleardoublepage
\thispagestyle{empty}

\begin{center}
{\large\bfseries Rigidez de ovaloides y visualización computacional}\\
\end{center}
\begin{center}
Eva María González García\\
\end{center}


\vspace{0.7cm}
\noindent{\textbf{Resumen}}\\

En este trabajo se estudian las superficies rígidas, esto es, aquellas superficies de $\mathbb{R}^3$ en las que cualquier isometría suya es la restricción de un movimiento rígido de $\mathbb{R}^3$. El principal resultado es el teorema de Cohn-Vossen, en el que se demuestra que los ovaloides de $\mathbb{R}^3$ son superficies rígidas.
${ }$\\

También se ha estudiado e implementado un algoritmo de visualización como es Ray-marching el cual mediante rayos lazados desde algún punto y algoritmos iterativos de aproximación de soluciones como son Newton-Raphson y Regula-Falsi nos permite visualizar superficies representadas de forma computacional mediante su ecuación implícita y se han visualizado algunos ejemplos de ovaloides mediante este procedimiento.


\vspace{0.7cm}
\noindent{\textbf{Palabras clave}: superficie rígida, ovaloides, movimiento rígido, isometría, ray-marching, ray-tracing, newton-raphson, regula-falsi, ecuación implícita, superficie.}\\
\cleardoublepage


\thispagestyle{empty}


\begin{center}
{\large\bfseries Rigidity of ovaloids and computational visualization}\\
\end{center}
\begin{center}
Eva María González García\\
\end{center}

\vspace{0.7cm}
\noindent{\textbf{Abstract}}\\


Seeing surfaces as figures which bend smoothly in Euclidean space have no corners or edges and do not intersect themselves, we can define intuitively surfaces like a subset of $\mathbb{R}^3$ such that each of its points has a neighbourhood similar to a piece of a plane which bends smoothly and without self-intersections when bent. We can see others definitions of surfaces but this is with which we are going to work.
${ }$\\

In this project we want to talk about the rigidity of surfaces, more cocretely the rigidity of the ovaloids. In this way, we will see the main theorem of this project, the theorem of ovaloid rigidity by Cohn-Vossen. For which we will need to know the notion of rigid movement, isometry, rigidity, curvature, first and second fundamental form... After, we will program a computational visualization algorithm with which we will be able to visualize surfaces expressed in implicit equations and we will visualize some examples of ovaloids.
${ }$\\

In previous results, we could see that we could have a sence of the shape near of a given point  studying the curvatures of all the curves which pass through this point in order to know how the surface bends in space. But we also saw that all of the curvatures of these curves are given by a quadratic form on the tangent plane takes. This quadratic form of which we speak is called as the second fundamental form. This quadratic form can be seen as a diagonalizable matrix, since it is an endomorphism of a Euclidean vector space, which alows us to define the two principal curvatures which are calculated as the eigenvalues of this matrix. This principal curvatures could be called as the maximum and the minimum curvature in a given point.
${ }$\\

To know how the shape of the surface is in a neighbourhood near of a given point the Gauss map $N : S \to \mathbb{S}^2$ is used to know how the normal vector of a surface varies when we move from a point to points close to it. In this case of the surfaces, the variation will be controlled by the differential of the Gauss map $(dN)_p : T_p S \to T_p S$.
${ }$\\

From the endomorphism $(dN)_p$ of $T_pS$, $p \in S$, we could define a bilinear form $\sigma_p$:

$$ \sigma_p : T_pS \times T_pS \to \mathbb{R}, \;\;\;\; p \in S, $$
$$ \sigma_p(v,w) = - \langle (dN)_p(v), w \rangle, \;\;\;\; v,w \in T_pS. $$
And we called this form as the second fundamental form of the surface $S$ at the point $p$, which will be very useful in the proofs of this project.
${ }$\\

An ovaloid is a surface such that its Gauss curvature, this is de determinant of $dN$, is always positive. Therefore, its second fundamental form is always positive.
${ }$\\

These ovaloids have the property that for each of its points the surface is entirely to one side of the tangent plane at that point. In addition the ovaloids are rigid surfaces this is the point that is wanted to prove in this first part of the project. We must know that a rigid surface is one that for any isometry, defined on that surface, is a restriction of a rigid motion. The rigidity of the ovaloids is proved in the Cohn-Vossen Theorem, which was demonstrated in 1927.
${ }$\\

The first section is an introduction in which several concepts necessary to understand the rigidity of the ovaloids are discussed, such as the definitions of isometry (application that preserves the length of the curves), rigid motion (application that conserves distances), rigidity and ovaloid, as well as other tools and results, such as Gauss curvature and media and the main curvatures, tubular neighborhoods, integration results that will be used, etc.
${ }$\\

Further on, it will be seen that an isometry that retains the second fundamental form is the restriction of a rigid motion. And its reciprocal will also be seen, a rigid motion is the extension of an isometry that preserves the second fundamental form, this will be seen in the fundamental theorem of the theory of the local surface. These two theorems tell us that it differentiates a rigid motion from an isometry.
${ }$\\

Next, we will see Gauss's well-known theorem Egregium, and its proof, this theorem says that the curvature of Gaus is invariant by isometries and from it, we can extract that the curvature of Gauss does not depend on the way in which the surface is bends in $\mathbb{R}^3$.
${ }$\\

It will be seen, the proof of the rigidity of the sphere. As it is easy to observe, the sphere is an example of ovaloid. This proof will be done using the Gaussian Egregium Theorem. And to complete this first part of the project, the Herglotz integral Forum will be built, this integral is an important tool that will be used to prove the theorem of the rigidity of the ovaloids of Cohn-Vossen.
${ }$\\

In the second part of this project, the visualization of surfaces will be discussed, the purpose of this part is to see how to visualize nontrivial surfaces given in their implicit equations. 
${ }$\\

The surfaces we intend to visualize are ovaloid. In spite of the algorithm serves for any surface that is expressed in its implicit equation.
${ }$\\

There are other possible ways to visualize surfaces, for example the rasterization method is a very fast one that allows to make changes in the scene in real time. On the contrary, Ray-tracing, which is seen in chapter 4, needs a lot of time to get the images and does not allow it to be used for video games which need a response in real time. On the other hand, Ray-tracing gives much more realistic results since it allows to calculate light effects such as reflection and reflaction and the images become indistinguishable from photographs and videos taken from real life. In this project, Ray-tracing is used because the rasterization method would involve rendering the ovaloids using triangle meshes, while with Ray-tracing we can use their implicit equations. Apart from the rasterization method, being the representation by triangles occupies more space while Ray-tracing only needs the implicit equation.
${ }$\\

In a first chapter some things necessary for visualization will be prepared and in the next chapter we will see the approach methods of solutions applied to the visualization of scenes.
${ }$\\

It will be seen what is needed for the representation of surfaces, in this way we will see how to save the information of the image of the scene. This scene will be saved in a matrix with the same length in rows and columns that rows and columns of pixels the window in which it is going to visualize has. We will also see how the color will be represented by the RBG agreement and later a pinhole camera will be programmed with which we will visualize the scene. To visualize surfaces, rays will be used that will be represented computationally by their mathematical equations, just like the surfaces. These rays will be launched from the observer's area and it will be checked if it intersects the surface, if so, the corresponding color is calculated for the point in the intersection. To prove that everything we have programmed works correctly, we will try the program with simpler figures that do not need to be treated by their equations implicitly. And you will see how to calculate the intersection of these figures with the rays. Specifically, the figures that will be seen are the sphere and the cube aligned with the axes. These figures do not need the use of iterative methods of approaching solutions.
${ }$\\

We will use the computational representation of rays (half lines) that start from the point where "the observer" is. These rays are associated to each of the pixels of the window where the scene will be displayed. If the ray associated with a certain pixel intersects the surface, the color of the surface at that point will be calculated taking into account the lighting and the color of the surface, and the calculated color will be projected on the pixel in question, that is, it will be the color that the pixel will take. Thus, pixel by pixel, the image of the scene we will visualize will be created.
${ }$\\

Next, we will see what approach methods of iterative solutions are going to be used, these are the Newton-Raphson method and the Regula-Falsi method. Each one has its advantages and disadvantages, for example one is faster, but the other ensures convercia when the other does not. For this reason, a combined use of both methods will be used. We will compare these methods by testing their convergence orders are 2 and 1 respectively and we will visualize several examples of ovaloids given in implicit equations.
${ }$\\

Some concepts of convergence of solutions will be seen, such as the order of convergence that indicates the speed with which the sequence approaches the solution.
${ }$\\

The main sources of this work have been $Curves$ $and$ $surfaces$ of S. Montiel and A. Ros for the rigidity of ovaloids and $Realistic$ $ray-tracing$ of Peter Shirley for the visualization of surfaces.
${ }$\\


\vspace{0.7cm}
\noindent{\textbf{Keywords}: rigid surface, ovaloids, rigid motion, isometry, ray-marchig, ray-tracing, newton-raphson, regula-falsi, implicit equation, surface.}\\

\chapter*{}
\thispagestyle{empty}

\noindent\rule[-1ex]{\textwidth}{2pt}\\[4.5ex]

Yo, \textbf{Eva Mª González García}, alumno de la titulación DOBLE GRADO EN INGENIERÍA INFORMÁTICA Y MATEMÁTICAS de la \textbf{Escuela Técnica Superior
de Ingenierías Informática y de Telecomunicación de la Universidad de Granada}y la \textbf{Facultad de ciencias de la Universidad de Granada}, con DNI XXXXXXXXX, autorizo la
ubicación de la siguiente copia de mi Trabajo Fin de Grado en la biblioteca del centro para que pueda ser
consultada por las personas que lo deseen.

\vspace{6cm}

\noindent Fdo: Eva Mª González García

\vspace{2cm}

\begin{flushright}
Granada a X de mes de 2018.
\end{flushright}


\chapter*{}
\thispagestyle{empty}

\noindent\rule[-1ex]{\textwidth}{2pt}\\[4.5ex]

D. \textbf{Francisco Urbano Pérez-Aranda}, Profesor del Área de Geometría y Topología del Departamento de Geometría y Topología de la Universidad de Granada.

\vspace{0.5cm}

D. \textbf{Carlos Ureña Almagro}, Profesor del Área de Lenguajes y Sistemas Informático del Departamento de Lenguajes y Sistemas Informáticos de la Universidad de Granada.


\vspace{0.5cm}

\textbf{Informan:}

\vspace{0.5cm}

Que el presente trabajo, titulado \textit{\textbf{Rigidez de ovaloides y visualización computacional}},
ha sido realizado bajo su supervisión por \textbf{Eva Mª González García}, y autorizamos la defensa de dicho trabajo ante el tribunal que corresponda.

\vspace{0.5cm}

Y para que conste, expiden y firman el presente informe en Granada a X de mes de 2018 .

\vspace{1cm}

\textbf{Los tutores:}

\vspace{5cm}

\noindent \textbf{Francisco Urbano Pérez-Aranda \ \ \ \ \ Carlos Ureña Almagro}

%\chapter*{Agradecimientos}
%\thispagestyle{empty}

%       \vspace{1cm}


%Poner aquí agradecimientos...

